\documentclass[
    12pt,
    a4paper,
    portuguese
]{abntex2}
\usepackage{lmodern}
\usepackage[T1]{fontenc}
\usepackage[utf8]{inputenc}
\usepackage{indentfirst}
\usepackage{color}
\usepackage{graphicx}
\usepackage{microtype}

\usepackage[brazilian, hyperpageref]{backref}
\usepackage[alf]{abntex2cite}

\renewcommand{\backrefpagesname}{Citado na(s) página(s):~}
% Texto padrão antes do número das páginas
\renewcommand{\backref}{}
% Define os textos da citação
\renewcommand*{\backrefalt}[4]{
	\ifcase #1 %
		Nenhuma citação no texto.%
	\or
		Citado na página #2.%
	\else
		Citado #1 vezes nas páginas #2.%
	\fi}%

\title{Ogof e o templo das joias}
\author{Pedro Freme \\ Marina Araujo \\ Eric Santos}
\local{São Caetano do Sul}
\data{2019}
\orientador{Teste \and Teste}
\instituicao{%
    FATEC São Caetano do Sul
    \par
    Tecnológico em Jogos Digitais
}

\tipotrabalho{TCC}

\preambulo{Colocar o preambulo aqui}

\makeatletter
\hypersetup{
     	%pagebackref=true,
		pdftitle={\@title}, 
		pdfauthor={\@author},
    	pdfsubject={\imprimirpreambulo},
	    pdfcreator={LaTeX with abnTeX2},
		pdfkeywords={abnt}{latex}{abntex}{abntex2}{trabalho acadêmico}, 
		colorlinks=true,       		% false: boxed links; true: colored links
    	linkcolor=blue,          	% color of internal links
    	citecolor=blue,        		% color of links to bibliography
    	filecolor=magenta,      		% color of file links
		urlcolor=blue,
		bookmarksdepth=4
}
\makeatother

\makeatletter
\setlength{\@fptop}{5pt} % Set distance from top of page to first float
\makeatother


\newcommand{\quadroname}{Quadro}
\newcommand{\listofquadrosname}{Lista de quadros}

\newfloat[chapter]{quadro}{loq}{\quadroname}
\newlistof{listofquadros}{loq}{\listofquadrosname}
\newlistentry{quadro}{loq}{0}

% configurações para atender às regras da ABNT
\setfloatadjustment{quadro}{\centering}
\counterwithout{quadro}{chapter}
\renewcommand{\cftquadroname}{\quadroname\space} 
\renewcommand*{\cftquadroaftersnum}{\hfill--\hfill}

\setfloatlocations{quadro}{hbtp}

% O tamanho do parágrafo é dado por:
\setlength{\parindent}{1.3cm}

% Controle do espaçamento entre um parágrafo e outro:
\setlength{\parskip}{0.2cm}  % tente também \onelineskip

% ---
% compila o indice
% ---
\makeindex


\begin{document}

\selectlanguage{brazil}

% Retira espaço extra obsoleto entre as frases.
\frenchspacing 

\imprimircapa


\begin{dedicatoria}
   \vspace*{\fill}
   \centering
   \noindent
   \textit{ Este trabalho é dedicado às crianças adultas que,\\
   quando pequenas, sonharam em se tornar cientistas.} \vspace*{\fill}
\end{dedicatoria}

\begin{agradecimentos}

Escrever

\end{agradecimentos}

% ---
% RESUMOS
% ---

% resumo em português
\setlength{\absparsep}{18pt} % ajusta o espaçamento dos parágrafos do resumo
\begin{resumo}
Lalalala

 \textbf{Palavras-chave}: latex. abntex. editoração de texto.
\end{resumo}

% resumo em inglês
\input{abstract.tex}

% ---
% inserir lista de ilustrações
% ---
\pdfbookmark[0]{\listfigurename}{lof}
\listoffigures*
\cleardoublepage
% ---

% ---
% inserir lista de quadros
% ---
\pdfbookmark[0]{\listofquadrosname}{loq}
\listofquadros*
\cleardoublepage
% ---

% ---
% inserir lista de tabelas
% ---
\pdfbookmark[0]{\listtablename}{lot}
\listoftables*
\cleardoublepage
% ---

% ---
% inserir lista de abreviaturas e siglas
% ---
\begin{siglas}
  \item[Modelo] Fazer
\end{siglas}
% ---

% ---
% inserir lista de símbolos
% ---
\begin{simbolos}
  \item[$ \Gamma $] Letra grega Gama
\end{simbolos}
% ---

% ---
% inserir o sumario
% ---
\pdfbookmark[0]{\contentsname}{toc}
\tableofcontents*
\cleardoublepage
% ---



% ----------------------------------------------------------
% ELEMENTOS TEXTUAIS
% ----------------------------------------------------------
\textual

% \begin{figure}[htb]
% 	\caption{\label{fig_grafico}Gráfico produzido em Excel e salvo como PDF}
% 	\begin{center}
% 	    \includegraphics[scale=0.5]{imagens/Rackham_elves.jpg}
% 	\end{center}
% 	\legend{Fonte: \citeonline[p. 24]{araujo2012}}
% \end{figure}

\chapter{Visão geral}
\section{Nome do Jogo}
Ogof e o templo das joias.

\section{\textit{High Concept} do Jogo}
Ogof e o templo das joias é um game de fantasia, em que você assumirá o papel de 3 diferentes figuras, e trabalhar em conjunto para progredir.
% Ogof e o templo das joias é um game de fantasia, exploração e puzzle em que você irá assumir o papel de 3 figuras de diferentes épocas do tempo, 
% % cada qual com suas forças e fraquezas, para progredir será necessário explorar as forças de cada personagem
% e trabalhar em conjunto para descobrir os planos de Ogof.

% Conceito em 150 caracteres


\section{Gênero}

O jogo terá elementos de fantasia, exploração e puzzle....

% Descreva e justifique o genero do jogo

\section{Púbico Alvo}

Descreva e justifique o publico alvo do jogo
Adolescentes a partir de 16 anos....

\section{\textit{Game Flow}}

Ilustração ou tabela que demonstre todas as telas que o jogo terá e como se relacionam entre si

\section{Estilo estético}

Resumo

\section{Inspirações}
Apresente nesta seção jogos que foram utilizados pela equipe como inspiração na elaboração da ideia e/ou desenvolvimento do jogo.
Mostrar quais foram suas inspiraçÕes poderá ajudar o leitor a entender mais a respeito de seu jogo

\subsection{Outlander}
\subsection{Filha de Feiticeira}
\subsection{Desencanto}
\subsection{Os Vingadores - Guerra ininita}
\subsection{Franquia God of War}
\subsection{Sonhos de uma noite de verão}
\subsection{Brownies}
Brownies são pequenos seres noturnos do folclore britânico, que cuidam dos afazeres da casa, se ofendem facilmente e evitam serem vistos pelos humanos, traços de personalidades que utilizamos como inspiração para a formulação do Duende.
Todo: Inserir as referências na bibliografia, estou tendo dificuldades para achar os livros no formato digital.

\subsection{Alux}
Alux são pequenas criaturas da mitologia de alguns povos Mayas, costumeiramente são invisíveis, mas podem se tornar visíveis para se comunicar e assutar humanos.  São responsáveis pela boa colheita, espantando animais e controlando a chuva. 

\section{Equipe de Desenvolvimento}
É importante destacar que participação cada integrante da equipe tem ou teve no desenvolvimento do projeto.
Caso haja itens do projeto que não venham a ser desenvolvidos pelos integrantes da equipe, como por exemplo assets de áudio ou imagem que sejam baixados de bancos gratuitos, é importante relatar essa situação aqui para que se tenha uma noção exata da autoria dos vários componentes do projeto.

A equipe dividiu-se de forma a otimizar as forças de cada um dos integrantes, desta forma ficamos com a seguinte divisão de tarefas não rigidas. 

Marina atuou fortemente com o game design, character design e história. 

Pedro atuou como programador e 

Eric atuou como arte técnica, animação e programação

\chapter{Gameplay e Mecânicas}
\section{\textit{Gameplay}}
\begin{enumerate}
\item Duende
\end{enumerate}
\begin{itemize}
\item Maçã:
\end{itemize}
Quando uma maçã é coletada ela pode ficar guardada para um momento de conveniência. Quando acionadende fica invisivel por 5 segundos.
\begin{itemize}
\item Banana:
\end{itemize}
Segue os mesmos princípios de coleta das maçãs. Quando acionadas causam chuvas que podem ser ácidas, para afastar inimigos, ou férteis fazendo crescer árvores e vegetações usadas em beneficio do jogador, tais como pontes, troncos onde ele pode se esconder, etc.

\section{Progressão do Jogo}

Descreva como o jogo evolui e o que o jogador precisa fazer para progredir no jogo.

\section{Estrutura de Missões/Desafios}

Descreva os desafios micros e macros dentro de cada fase do jogo.

\section{Objetivos}

Descreva os desafios micros e macros dentro de cada fase do jogo.

\section{Mecânicas}

Descreva as regas do jogo (implícitas e explícitas) e como as partes (personagens do jogo) interagem entre si.

\section{Movimentação / Física}

Descreva como funcionará as movimentações dos personagens dentro jogo (as limitações e possibilidades) e como será o sistema de física.

\section{Objetos}

Descreva os objetos que existem no jogo e como o jogador interage com eles.

\section{Ações}

Descreva como o jogador interage de forma geral (controles, botões, objetos) com o mundo do jogo.

\section{Combate}

Descreva como o jogador interage de forma geral (controles, botões, objetos) com o mundo do jogo.

\section{Economia}

Descreva como funciona o sistema financeiro (moedas) e como elas são utilizadas no mundo do jogo.

\section{Opções do Jogo}

Descreva quais são as modalidades de jogo e como elas afetam/alteram o jogo.

\section{Salvar \& Replay}

Descreva o funcionamento do sistema de save ou autosave e se há modo New Game Plus.

\section{Easter Eggs, Cheats, bônus}

Descreva se o jogo possuir conteúdos escondidos, trapaças ou conteúdos extras (roupas, modos, etc.).

\chapter{Arte do Game}

\section{Elementos Visuais}

Descreva os elementos chave; como estão sendo desenvolvidos; qual o estilo. Inclua detalhes sobre a direção de arte, paleta de cores e inspirações.

\subsection{Inspirações}
Todos os personagens foram baseados em estatuetas artesanais, desenhados em 2D e por fim modelados em 3D. 
\subsection{Duende}
\begin{figure}[htb]
	\caption{\label{duendeRef}duendeRef}
	\begin{center}
	    \includegraphics[width=\textwidth]{imagens/duendeRef.jpg}
	\end{center}
	\legend{Fonte: Mandala de Luz - https://www.mandaladeluz.com.br/pumy-duende-amigo-chapeu-verde-8cm}
\end{figure}



\begin{figure}[htb]
	\caption{\label{duendePos}duendePos}
	\begin{center}
	    \includegraphics[width=\textwidth]{imagens/duendePosições.jpeg}
	\end{center}
	\legend{Fonte: Autoria Própria - Marina Araujo}
\end{figure}



\subsection{Ogof}
\begin{figure}[htb]
	\caption{\label{mago}mago}
	\begin{center}
	    \includegraphics[width=\textwidth/2]{imagens/mago.jpg}
	\end{center}
	\legend{Fonte: Macocaya - https://macocaya.es/es/varios/3134-mago-con-bola.html}
\end{figure}

\begin{figure}[htb]
	\caption{\label{Ogof}Ogof}
	\begin{center}
	    \includegraphics[width=\textwidth/2]{imagens/Ogof.jpg}
	\end{center}
	\legend{Fonte: Autoria Própria - Marina Araujo}
\end{figure}

\subsection{Paleta de cores}
\begin{figure}[htb]
	\caption{\label{paleta}paleta}
	\begin{center}
	    \includegraphics[width=\textwidth/2]{imagens/paleta.jpg}
	\end{center}
	\legend{Fonte: Própria Autoria}
\end{figure}

\section{Elementos Sonoros}

Descreva os elementos chave; como estão sendo desenvolvidos; qual o estilo musical. Inclua detalhes sobre efeitos sonoros e inspirações.

\chapter{Narrativa, Ambientação e Personagens}

\section{História e Narrativa}

[Descreva os elementos chave; como estão sendo desenvolvidos; qual o estilo musical. Inclua detalhes sobre efeitos sonoros e inspirações.]
O mago Ogof, sedento por poder, capturou três seres místicos, em três diferentes épocas para o seu ritual ardiloso. Quando os planetas se alinham ele captura o gato de uma bruxa, a esposa de um elfo e o filho de um xamã. O ritual espalhou pelo mundo 3 joias que guardam as almas dos raptados. Resgate as joias com esses três personagens jogáveis, mas o ritual só é reversível com as três joias juntas.

\section{Áreas do Jogo}

\subsection{O mundo de Ogof}
Mesopotâmia, idade antiga. Ogof vive na cidade de Ur na época de 4000 a.C. à 1900 a.C. Uma sociedade hidráulica onde todo o estilo de vida gira em torno de seus rios Tigre e Eufrates. Para tal, desenvolveram grandes habilidades arquitetônicas para lidar com as enchentes . 

A sociedade residente de Ur é estamental, isto é, pessoas nascidas em uma determinada classe social morrerão nessa mesma classe e sua política tem bases religiosas. Isso significa que um líder político é necessariamente alguém que se diz parente ou conhecido de algum deus.

\subsection{O mundo dos Duendes}
Baseada em Geiranger, Noruega - 1920
Situado em um belo fiorde, suas terras são afastadas dos outros lugares pelas montanhas rochosas que a cercam, e a forma mais fácil de chegar a essas terras é por um navegante que tenha habilidade suficiente para navegar pelo mar através dos
fiordes, o que não é tarefa fácil. Por essa razão duendes são seres desconhecidos e ocultos que transitam com facilidade pelo mundo dos humanos e os ajudam se julgarem que tem boa índole, do contrário podem atrapalhar muito suas vidas escondendo coisas, assustando animais e fazendo ruídos assustadores. Em ambos os casos duendes nunca devem ser vistos.

Algumas histórias de ninar como o "Conto do Sapateiro" narram a história de humanos que descobriram os duendes e, cheios de gratidão, lhe presentearam com roupas e assistiram satisfeitos a sua partida acreditando que os duendes haviam encerrado o seu trabalho e partiram felizes. Mas, para a sociedade destes seres misticos a regra é clara: receber roupas significa que você foi descoberto e desonrou as tradições do seu povo, a sentença é o banimento. 

\subsection{O mundo da Bruxa}
\subsection{O mundo do Xamã}


\section{Personagens}
\subsection{Ogof}
\textbf{Back history: }Sedento por conhecimento, acessou ensinamentos ocultos que desestabilizaram sua mente e agora o poder é sua obsessão.

\textbf{Físico:} velho, parrudo, imponente. Com uma barba longa e grisalha.

\textbf{Descrição psicológica:} Impiedoso e insano, ocasionalmente fala coisas sem sentido.

\textbf{Arquétipo:} Sombra

\subsection{Duende}
\textbf{Back history:} Trabalhador com uma rotina tediosa e repetitiva. Seu mal humor é apenas curado por sua esposa doce.

\textbf{Descrição física:} Pequeno, gordo e mal vestido.

\textbf{Descrição psicológica:} Temperamental, revoltado e mau humorado.

\textbf{Arquétipo:} Pícaro

\textbf{Temperamento:} Irritadiço quando está com o contador de vida cheio. Quando o contador está próximo do fim fica furioso.

\textbf{Iteração: } Neste universo, duendes são seres misticos que coexistem com humanos. Parte dos desafios é manter-se despercebido. São mais facilmente vistos por crianças.

\textbf{Habilidades: }
\begin{itemize}
\item  Invisibilidade: Auxilia na furtividade do personagem mas depende diretamente dos recursos que o jogador possui. Ao longo da fase deverá coletar maçãs que encherão uma barra de energia que, quando completa, libera a invisibilidade. O jogador deve gerenciar a barra para usá-la em obstáculos estratégicos.
\end{itemize}


\begin{itemize}
\item Chuva:  funciona da mesma forma que a invisibilidade, ao coletar uma quantidade de bananas suficiente para encher a barra poderá invocar chuva, mas deverá administrar os recursos com cautela.
\end{itemize}
Ambos os itens, maçãs e bananas são adquiridos tanto buscando-os em esconderijos no cenário quanto prestando favores aos humanos.

\textbf{Fraqueza:} De acordo com [N˜AO LEMBRO QUEM DISSE ISSO] duendes são seres desatentos e facilmente distraíveis. Para simular essa característica, o jogador só pode enxergar com clareza poucos metros de distância a partir do duende. Ao restante do cenário foi adicionado um desfoque para que seja levemente borrado, dificultando a identificação dos elementos.

\subsection{Bruxa}
\begin{figure}[htb]
	\caption{\label{fig_grafico}concept Bruxa}
	\begin{center}
	    \includegraphics[scale=0.5]{imagens/capa portifolio.jpeg}
	\end{center}
	\legend{Fonte: Própria Autoria(Marina Araujo)}
\end{figure}

\textbf{Back history:} Estudiosa e conhecedora das artes do oculto. Respeita as artes mágicas e não faz uso sem autorização. Treinada pelas anciãs.

\textbf{Descrição física/psicologica:} Metamorfa, sua personalidade é afetada pela energia do ambiente. Quando suas energias são boas e prestativas assume a aparência de bruxa má. Quando a escuridão a consome se torna uma mulher bela e jovem.

\textbf{Arquétipo: }Camaleão

\subsection{Xamã}

\textbf{Back history: }indígena norte americano, sábio e orientador da tribo.

\textbf{Descrição física: }Pele vermelha, e aparência respeitável. Inspira confiança mas intimida. Utiliza pele de lobo na cabeça.

\textbf{Descrição psicológica:} Sério e protetor como um grande pai. Transcende o corpo e trabalha com espíritos para proteger o grupo.

\textbf{Arquétipo:} Mentor

\section{Fases}

\begin{enumerate}
\item A redenção de um duende - Fullkomnun: 
\end{enumerate}
O duende chega a tempo de ver Ogof levando sua esposa e o persegue até conseguir arrancar-lhe um pedaço da roupa, mas não consegue impedí-lo. O mestre da vila se aproxima e encontra o duende enfurecido. Percebendo que ele tem uma pedaço de roupa nas mãos,  manda que encontre o ancião a procura de ajuda. 
O ancião então lhe diz que, para abrir o portal pelo qual Ogof fugiu serão necessários itens humanos [A DEFINIR] e itens de agricultura. O duende então sai em busca destes itens.

	4.1.1 Pontos de interesse
    	4.1.1.1 O pico do fiorde
        	O  ancião vive no ponto mais alto do reino: o topo de um icônico fiorde. Sob o ar rarefeito e a densidade da nevoa, apenas os duendes mais resistentes e determinados conseguem chegar a sua morada e assim receber como recompensa as suas sábias orientações.
            
        4.1.1.2 A plantação de Elvdans
        	 A cidade de Elvdans é povoada por camponeses humanos, um povo simples e supersticioso, porém amigável. O jogador deverá encontrar uma plantação suntuosa na cidade, onde uma família de duendes deixara itens mágicos para trazer prosperidade. Os fazendeiros não necessitam mais delas, você deve levá-las consigo para atender a ordem do ancião e posteriormente deixa-los a vista para que outro humano os pegue. Duendes são uma especie temperamental e emotiva por isso tem tendências a ajudar pessoas ao longo de suas viagens. Quando o usuário interromper seus afazeres para auxiliar alguém receberá itens que lhe serão uteis mais tarde. 


\section{Fase de Treino e/ou Tutorial}

Descreva como funciona a fase de treino ou o tutorial do jogo que aparece na tela.


\chapter{Interface}

\section{Sistema Visual}

Descreva como funciona a fase de treino ou o tutorial do jogo que aparece na tela.


\section{Sistema de Controle}

Descreva como o jogador controla o jogo e como são os comandos específicos.

\section{Sistema de Ajuda}

Descreva (se houver) o sistema de dicas para o jogador caso ele não consiga continuar o jogo.

\chapter{Inteligência Artificial (AI)}
Esta sessão será preenchida no próximo semestre.

\section{Oponentes e AI Inimiga}

Para os oponentes pretende-se utilizar a técnica de Circulo de batalha explicada pelo \citeonline{BattleCi} em seu artigo. No próximo semestre preencheremos esta sessão com maiores informações e explicações de suas aplicações.


\section{AI parceira ou não-inimigas}


\section{AI de Suporte}


\chapter{Aspectos Técnicos}

\section{Plataforma de Produção}

Descreva e justifique para quais plataformas o jogo está sendo produzido (PC, Android, consoles etc).

\section{Hardware e Software de Desenvolvimento}

Descreva e justifique para quais plataformas o jogo está sendo produzido (PC, Android, consoles etc).

\section{Requerimentos de Rede}

Descreva de que forma o jogo fará uso da Internet, caso utilize.

\chapter{Cronograma}

\section{Cenário Atual}

Até o presente momento, decidiu-se a história geral do jogo, que se desenvolve em torno dos três personagens principais, Duende, Bruxa e Xamã, bem como o antagonista Ogof.

Para que o jogador tenha um maior relacionamento com os personagens, optou-se por apresenta-los em seus ambientes de origem, e nesses ambientes demonstrar o uso das habilidades únicas de cada um deles, com isso definiu-se de forma geral as três primeiras fases, e suas palestas de cores, que foram baseadas em locais reais. 

Definimos mecânicas gerais para os personagens, bem como as mecânicas específicas dos personagens Duende e Xamã, três cada um.

As inspirações, e referências levaram a opção \textit{low poly} como estilo estético que será aplicado no jogo. 


\section{Próximos Passos}

Para os próximos passos do projeto, será feito uma lapidação e maior definição das características e habilidades da personagem Bruxa.

Será iniciada a parte técnica de execução do projeto, iniciando-se pela modelagem 3D Low Poly dos personagens. Espera-se que seja empenhada 2 semanas para cada personagem, bem como sua animação, com um dos personagens em mão será iniciada a implementação de mecânicas e a programação necessária.

A partir da segunda semana de implementação das mecânicas, serão feitas rodadas de testes e feedbacks com \textit{beta testers} quinzenalmente, para polimento e avaliação de diversão.

O detalhamento destas atividades pode ser observado na figura \ref{fig_crono} - Cronograma

\begin{figure}[!htb] \caption{\label{fig_crono}Cronograma} \begin{center}
\includegraphics[width=\textwidth]{imagens/crono.png} \end{center}
\legend{Fonte: Autoria Nossa} \end{figure}

\chapter{Modelo de Negócio}
Descreva nessa seção como pretende monetizar o jogo, se é que haverá monetização. Se for vendido, por exemplo, informe qual a faixa de preço. Se houver compras dentro do jogo, como serão feitas? Quais serão as estratégias de divulgação para ampliar a comunidade de fãs e possíveis compradores (redes sociais, mídia especializada, demo, early access etc.).
Mesmo que não haja monetização no seu jogo, nesta seção devem ser descritos os elementos definidos pelo(a) orientador(a) responsável pela disciplina Empreendedorismo, sendo neste momento o canvas comentado e a análise SWOT do jogo.




\postextual

\bibliography{00_bibliografia}

% ---
% Inicia os apêndices
% ---

\cite{villanuelva2014concepcion,briggs1957english,storniolo2009out,grimm2003elves,bane2013encyclopedia,britannica_2011, carolyn_2016,lyneis1995virgin,kohler2008mesa, classificacao, heyder1997anasazi}
\cite{godofwar, dessencanto, irish2005game}

\begin{apendicesenv}

% Imprime uma página indicando o início dos apêndices
\partapendices

% ----------------------------------------------------------
\chapter{Ogof}
% ----------------------------------------------------------

Houve tempo em que os Deuses ainda andavam sobre a Terra, em que humanos e celestiais conviviam em harmonia. Alguns eram escolhidos para despertar suas habilidades mais escondidas, e assim ajudar aqueles que lhes ensinaram a continuar seus trabalhos. 

Mas o orgulho abateu a sociedade, e por muitos anos humanos escondiam o ressentimento de não fazer parte dos escolhidos, aos poucos esse ressentimento foi crescendo e tornou-se um misto de inveja e ira. Tendo em vista os futuros atos destes humanos, os deuses decidiram que estariam mais a salvo em outro plano, e apenas aqueles apontados por eles teriam o conhecimento para acessar esse novo plano. Assim incumbiram humanos de confiança manter a ordem e definir seus sucessores e estudiosos. 

Então chega a era em que os Deuses exilam-se da Terra, mas ainda governam por seus escolhidos humanos. Ogof, ainda jovem, por sua extrema curiosidade, e dedicação àquilo que lhe encantava, foi escolhido para estudar junto com (Um nome aqui), que dedicava-se ao deus (nome aqui). 

Poe anos Ogof se dedicou aos estudos, e dentre todos ícones sobresaia-se, como se a magia dos deuses fosse algo natural, como se sempre estivesse consigo, e ainda assim sempre se aplicava aos estudos e a tentar entender cada vez mais o mundo em que habitava.

Percebendo que os deuses não mais andavam pela terra, alguns homens começaram a se organizar para retirar o poder daqueles que reinavam, acreditavam que precisavam se libertar dos deuses, e assim de seus escolhidos para que pudessem modificar seus destinos.

Ogof que agora é um grande mago, percebe isso, e agrupa outros grande ícones dos deuses para que possam manter a ordem perante o caos que está para se formar.

Por anos os icones conseguiram manter seu objetivo, e com a fartura que se manteve nesse tempo, as organizações que haviam se levantado contra os a ordem dos Deuses até se silenciaram. Mas devido a uma terrível perturbação na Terra, findou-se esse momento, e então não se viam mais festas em que mel e leite eram abundantes, as colheitas se reduziram, o gado emagreceu, e a desconfiança sobre os homens se instaurou, a violência como um rompente toma conta das cidades, ganancia e caos prosperam. E os ícones cada vez mais enfraquecidos pela sobrecarga de atenções que isso gerou.

Ogof convoca os maiores ícones, e decreta que precisam de ajuda, e então da-se o início das escolas de alta magia, não mais apenas um estudante por mestre, mas sim a transmissão do conhecimento para vários, com isso ele acredita que conseguirá retomar a controle, e voltar aos tempos passados, de pastos verdes, e que a paz reinava.

Durante este encontro Ogof sente-se incomodado com algo que não consegue afirmar o que possa ser, e isso o acompanha por todo discurso, volta caminhando pela noite desconfortavelmente, deixando se perder em pensamentos acreditando que deixou passar algum detalhes em algum momento. 

Ao chegar ao seu lar, cai de joelhos, o peito lhe aperta e a respiração fica difícil, a escuridão toma conta de sua visão, o olho enche d'agua, e grandes gotas começam a escorrer pelo seu rosto taciturno. No chão de sua sala, jaz sua amada esposa, toma-a em seu braço, e assim permanece por três dias. 

Ao acordar desta desilusão tudo é dor, e decide que há apenas uma forma de sanar esta angustia.

Ogof toma uma decisão...



\end{apendicesenv}
% ---


\end{document}
