\documentclass[
    12pt,
    a4paper,
    openany,
    portuguese
]{abntex2}
\usepackage{lmodern}
\usepackage[T1]{fontenc}
% \usepackage[utf8]{inputenc}
\usepackage{indentfirst}
\usepackage{color}
\usepackage{graphicx}
\usepackage{microtype}
\usepackage{tabularx}
\usepackage{titlecaps}
% \usepackage{float}
 \usepackage[footnotesize]{caption} 

% \usepackage[brazilian, hyperpageref]{backref}
\usepackage[alf,abnt-etal-text=default]{abntex2cite}

% \renewcommand{\backrefpagesname}{Citado na(s) página(s):~}
% Texto padrão antes do número das páginas
% \renewcommand{\backref}{}
% Define os textos da citação
% \renewcommand*{\backrefalt}[4]{
% 	\ifcase #1 %
% 		Nenhuma citação no texto.%
% 	\or
% 		Citado na página #2.%
% 	\else
% 		Citado #1 vezes nas páginas #2.%
% 	\fi}%
\newcommand{\citautoryear}[1]{\citeauthoronline{#1}, \citeyear{#1}}

\titulo{Ogof e O Templo das joias}
\autor{ Eric Santos\\ Marina Araujo \\Pedro Freme}
\local{São Caetano do Sul}
\data{2019}
%\orientador{Teste \and Teste}
\instituicao{%
    % FATEC São Caetano do Sul
    % \par
    % Tecnologia em Jogos Digitais
}
\preambulo{Trabalho de conclusão de curso submetido como requisito parcial para a conclusão do Curso Superior de Tecnologia em Jogos Digitais da Faculdade de Tecnologia de São Caetano do Sul - Antônio Russo orientados pelos Professores  Prof. Me. Miguel Marílio Saad Junior, Prof\textsuperscript{a} Dra. Raquel Silva e Prof\textsuperscript{a} Me. Leide Aparecida Vieira}

%\tipotrabalho{monografia}

% \titulo{Modelo Canônico de\\ Trabalho Acadêmico com \abnTeX}
% \autor{Equipe \abnTeX}
% \local{Brasil}
% \data{2014, v-1.9.2}
% \orientador{Lauro César Araujo}
% \coorientador{Equipe \abnTeX}
% \instituicao{%
%   Universidade do Brasil -- UBr
%   \par
%   Faculdade de Arquitetura da Informação
%   \par
%   Programa de Pós-Graduação}
% \tipotrabalho{Tese (Doutorado)}
% % O preambulo deve conter o tipo do trabalho, o objetivo, 
% % o nome da instituição e a área de concentração 
% \preambulo{Modelo canônico de trabalho monográfico acadêmico em conformidade com
% as normas ABNT apresentado à comunidade de usuários \LaTeX.}
% ---


\makeatletter
\hypersetup{
     	%pagebackref=true,
		pdftitle={\@title},
		pdfauthor={\@author},
    	pdfsubject={\imprimirpreambulo},
	    pdfcreator={LaTeX with abnTeX2},
		pdfkeywords={abnt}{latex}{abntex}{abntex2}{trabalho acadêmico},
		colorlinks=true,       		% false: boxed links; true: colored links
    	linkcolor=black,          	% color of internal links
    	citecolor=black,        		% color of links to bibliography
    	filecolor=magenta,      		% color of file links
		urlcolor=black,
		bookmarksdepth=4
}
\makeatother

\makeatletter
\setlength{\@fptop}{5pt} % Set distance from top of page to first float
\makeatother


\newcommand{\quadroname}{Quadro}
\newcommand{\listofquadrosname}{Lista de quadros}

\newfloat[chapter]{quadro}{loq}{\quadroname}
\newlistof{listofquadros}{loq}{\listofquadrosname}
\newlistentry{quadro}{loq}{0}

% configurações para atender às regras da ABNT
\setfloatadjustment{quadro}{\centering}
\counterwithout{quadro}{chapter}
\renewcommand{\cftquadroname}{\quadroname\space}
\renewcommand*{\cftquadroaftersnum}{\hfill--\hfill}

\setfloatlocations{quadro}{hbtp}

% O tamanho do parágrafo é dado por:
\setlength{\parindent}{1.3cm}

% Controle do espaçamento entre um parágrafo e outro:
\setlength{\parskip}{0.2cm}  % tente também \onelineskip

% ---
% compila o indice
% ---
\makeindex


\renewcommand{\thesubsubsection}{\alph{subsubsection}}
\renewcommand{\thesubsection}{\thesection.\Alph{subsection}}

\renewcommand{\ABNTEXchapterfontsize}{\normalsize}

\renewcommand{\ABNTEXsectionfontsize}{\normalsize}

\renewcommand{\ABNTEXsubsectionfontsize}{\normalsize}


% \renewcommand{\ABNTEXfontereduzida}{\small}

\begin{document}

\selectlanguage{brazil}

% Retira espaço extra obsoleto entre as frases.
\frenchspacing

\pretextual

\imprimircapa

\imprimirfolhaderosto


% \begin{dedicatoria}
%   \vspace*{\fill}
%   \centering
%   \noindent
%   \textit{ Este trabalho é dedicado às crianças adultas que,\\
%   quando pequenas, sonharam em se tornar cientistas.} \vspace*{\fill}
% \end{dedicatoria}

% \begin{agradecimentos}

Escrever

\end{agradecimentos}

% ---
% RESUMOS
% ---

% resumo em português
% \setlength{\absparsep}{18pt} % ajusta o espaçamento dos parágrafos do resumo
\begin{resumo}
Lalalala

 \textbf{Palavras-chave}: latex. abntex. editoração de texto.
\end{resumo}

% resumo em inglês
% \begin{resumo}[Abstract]
 \begin{otherlanguage*}{english}
   This is the english abstract.

   \vspace{\onelineskip}
 
   \noindent 
   \textbf{Keywords}: latex. abntex. text editoration.
 \end{otherlanguage*}
\end{resumo}

% ---
% inserir lista de ilustrações
% ---
\pdfbookmark[0]{\listfigurename}{lof}
\listoffigures*
\cleardoublepage
% ---

% ---
% inserir lista de quadros
% ---
\pdfbookmark[0]{\listofquadrosname}{loq}
\listofquadros*
\cleardoublepage
% ---

% ---
% inserir lista de tabelas
% ---
% \pdfbookmark[0]{\listtablename}{lot}
% \listoftables*
% \cleardoublepage
% ---

% ---
% inserir lista de abreviaturas e siglas
% ---
% \begin{siglas}
%   \item[Modelo] Fazer
% \end{siglas}
% ---

% ---
% inserir lista de símbolos
% ---
% \begin{simbolos}
%   \item[$ \Gamma $] Letra grega Gama
% \end{simbolos}
% ---

% ---
% inserir o sumario
% ---
\pdfbookmark[0]{\contentsname}{toc}
\tableofcontents*
\cleardoublepage
% ---



% ----------------------------------------------------------
% ELEMENTOS TEXTUAIS
% ----------------------------------------------------------
\textual

% \begin{figure}[htb]
% 	\caption{\label{fig_grafico}Gráfico produzido em Excel e salvo como PDF}
% 	\begin{center}
% 	    \includegraphics[scale=0.5]{imagens/Rackham_elves.jpg}
% 	\end{center}
% 	\legend{Fonte: \citeonline[p. 24]{araujo2012}}
% \end{figure}

\chapter{Visão geral}
\include{visaoGeral.tex}



\chapter{Gameplay e Macânicas}

Lorem ipsul dolort sit amet consetur

\section{\textit{Gameplay}}

Lorem ipsul dolort sit amet consetur

\section{Progressão do Jogo}

Lorem ipsul dolort sit amet consetur

\section{Estrutura de Missões/Desafios}

Lorem ipsul dolort sit amet consetur

\section{Objetivos}

Lorem ipsul dolort sit amet consetur

\section{Mecânicas}

Lorem ipsul dolort sit amet consetur

\section{Movimentação / Física}

Lorem ipsul dolort sit amet consetur

\section{Objetos}

Lorem ipsul dolort sit amet consetur

\section{Ações}

Lorem ipsul dolort sit amet consetur

\section{Combate}

Lorem ipsul dolort sit amet consetur

\section{Economia}

Lorem ipsul dolort sit amet consetur

\section{Movimentação / Física}

Lorem ipsul dolort sit amet consetur

\section{Opções do Jogo}

Lorem ipsul dolort sit amet consetur

\section{Salvar \& Replay}

Lorem ipsul dolort sit amet consetur

\section{Easter Eggs, Cheats, bônus}

Lorem ipsul dolort sit amet consetur


\chapter{Arte do Game}

\section{Elementos Visuais}

Lorem ipsul dolort sit amet consetur

\section{Elementos Sonoros}

Lorem ipsul dolort sit amet consetur

\chapter{Narrativa, Ambientação e Personagens}

\section{História e Narrativa}

Lorem ipsul dolort sit amet consetur

\section{Mundo do Jogo}

Lorem ipsul dolort sit amet consetur


\section{Visão Geral}

Lorem ipsul dolort sit amet consetur

\section{Áreas do Jogo}

Lorem ipsul dolort sit amet consetur


\section{Personagens}

\subsection{Ogof}
\subsection{Duende}
\subsection{Bruxa}
\subsection{Xamã}

Lorem ipsul dolort sit amet consetur

\section{Fases}

Lorem ipsul dolort sit amet consetur


\section{Fase de Treino e/ou Tutorial}

Lorem ipsul dolort sit amet consetur

\chapter{Interface}

\section{Sistema Visual}

Lorem ipsul dolort sit amet consetur


\section{Sistema de Controle}

Lorem ipsul dolort sit amet consetur

\section{Sistema de Ajuda}

Lorem ipsul dolort sit amet consetur

\chapter{Inteligência Artificial (AI)}


\section{Oponentes e AI Inimiga}

Lorem ipsul dolort sit amet consetur

\section{AI parceira ou não-inimigas}

Lorem ipsul dolort sit amet consetur

\section{AI de Suporte}

\chapter{Aspectos Técnicos}

\section{Plataforma de Produção}

\cite{shoemaker}
\cite{encyclopedia}
Lorem ipsul dolort sit amet consetur

\section{Hardware e Software de Desenvolvimento}

Lorem ipsul dolort sit amet consetur

\section{Requerimentos de Rede}

\chapter{Modelo de Negócio} 


\postextual

\bibliography{00_bibliografia}

% ---
% Inicia os apêndices
% ---

\cite{villanuelva2014concepcion,briggs1957english,storniolo2009out,grimm2003elves,bane2013encyclopedia,britannica_2011, carolyn_2016,lyneis1995virgin,kohler2008mesa, classificacao, heyder1997anasazi}
\cite{godofwar, dessencanto, irish2005game}

\begin{apendicesenv}

% Imprime uma página indicando o início dos apêndices
\partapendices

% ----------------------------------------------------------
\chapter{Ogof}
% ----------------------------------------------------------

Houve tempo em que os Deuses ainda andavam sobre a Terra, em que humanos e celestiais conviviam em harmonia. Alguns eram escolhidos para despertar suas habilidades mais escondidas, e assim ajudar aqueles que lhes ensinaram a continuar seus trabalhos. 

Mas o orgulho abateu a sociedade, e por muitos anos humanos escondiam o ressentimento de não fazer parte dos escolhidos, aos poucos esse ressentimento foi crescendo e tornou-se um misto de inveja e ira. Tendo em vista os futuros atos destes humanos, os deuses decidiram que estariam mais a salvo em outro plano, e apenas aqueles apontados por eles teriam o conhecimento para acessar esse novo plano. Assim incumbiram humanos de confiança manter a ordem e definir seus sucessores e estudiosos. 

Então chega a era em que os Deuses exilam-se da Terra, mas ainda governam por seus escolhidos humanos. Ogof, ainda jovem, por sua extrema curiosidade, e dedicação àquilo que lhe encantava, foi escolhido para estudar junto com (Um nome aqui), que dedicava-se ao deus (nome aqui). 

Poe anos Ogof se dedicou aos estudos, e dentre todos ícones sobresaia-se, como se a magia dos deuses fosse algo natural, como se sempre estivesse consigo, e ainda assim sempre se aplicava aos estudos e a tentar entender cada vez mais o mundo em que habitava.

Percebendo que os deuses não mais andavam pela terra, alguns homens começaram a se organizar para retirar o poder daqueles que reinavam, acreditavam que precisavam se libertar dos deuses, e assim de seus escolhidos para que pudessem modificar seus destinos.

Ogof que agora é um grande mago, percebe isso, e agrupa outros grande ícones dos deuses para que possam manter a ordem perante o caos que está para se formar.

Por anos os icones conseguiram manter seu objetivo, e com a fartura que se manteve nesse tempo, as organizações que haviam se levantado contra os a ordem dos Deuses até se silenciaram. Mas devido a uma terrível perturbação na Terra, findou-se esse momento, e então não se viam mais festas em que mel e leite eram abundantes, as colheitas se reduziram, o gado emagreceu, e a desconfiança sobre os homens se instaurou, a violência como um rompente toma conta das cidades, ganancia e caos prosperam. E os ícones cada vez mais enfraquecidos pela sobrecarga de atenções que isso gerou.

Ogof convoca os maiores ícones, e decreta que precisam de ajuda, e então da-se o início das escolas de alta magia, não mais apenas um estudante por mestre, mas sim a transmissão do conhecimento para vários, com isso ele acredita que conseguirá retomar a controle, e voltar aos tempos passados, de pastos verdes, e que a paz reinava.

Durante este encontro Ogof sente-se incomodado com algo que não consegue afirmar o que possa ser, e isso o acompanha por todo discurso, volta caminhando pela noite desconfortavelmente, deixando se perder em pensamentos acreditando que deixou passar algum detalhes em algum momento. 

Ao chegar ao seu lar, cai de joelhos, o peito lhe aperta e a respiração fica difícil, a escuridão toma conta de sua visão, o olho enche d'agua, e grandes gotas começam a escorrer pelo seu rosto taciturno. No chão de sua sala, jaz sua amada esposa, toma-a em seu braço, e assim permanece por três dias. 

Ao acordar desta desilusão tudo é dor, e decide que há apenas uma forma de sanar esta angustia.

Ogof toma uma decisão...



\end{apendicesenv}
% ---


\end{document}
