\documentclass[
    12pt,
    a4paper,
    portuguese
]{abntex2}
\usepackage{lmodern}
\usepackage[T1]{fontenc}
\usepackage[utf8]{inputenc}
\usepackage{indentfirst}
\usepackage{color}
\usepackage{graphicx}
\usepackage{microtype}

\usepackage[brazilian, hyperpageref]{backref}
\usepackage[alf]{abntex2cite}

\renewcommand{\backrefpagesname}{Citado na(s) página(s):~}
% Texto padrão antes do número das páginas
\renewcommand{\backref}{}
% Define os textos da citação
\renewcommand*{\backrefalt}[4]{
	\ifcase #1 %
		Nenhuma citação no texto.%
	\or
		Citado na página #2.%
	\else
		Citado #1 vezes nas páginas #2.%
	\fi}%

\title{Ogof e o templo das joias}
\author{Pedro Freme \and Marina Araujo \and Eric Santos}
\local{São Caetano do Sul}
\data{2019, v-0.0.1}
\orientador{Teste \and Teste}
\instituicao{%
    FATEC São Caetano do Sul
    \par
    Tecnológico em Jogos Digitais
}

\tipotrabalho{TCC}

\preambulo{Colocar o preambulo aqui}

\makeatletter
\hypersetup{
     	%pagebackref=true,
		pdftitle={\@title}, 
		pdfauthor={\@author},
    	pdfsubject={\imprimirpreambulo},
	    pdfcreator={LaTeX with abnTeX2},
		pdfkeywords={abnt}{latex}{abntex}{abntex2}{trabalho acadêmico}, 
		colorlinks=true,       		% false: boxed links; true: colored links
    	linkcolor=blue,          	% color of internal links
    	citecolor=blue,        		% color of links to bibliography
    	filecolor=magenta,      		% color of file links
		urlcolor=blue,
		bookmarksdepth=4
}
\makeatother

\makeatletter
\setlength{\@fptop}{5pt} % Set distance from top of page to first float
\makeatother


\newcommand{\quadroname}{Quadro}
\newcommand{\listofquadrosname}{Lista de quadros}

\newfloat[chapter]{quadro}{loq}{\quadroname}
\newlistof{listofquadros}{loq}{\listofquadrosname}
\newlistentry{quadro}{loq}{0}

% configurações para atender às regras da ABNT
\setfloatadjustment{quadro}{\centering}
\counterwithout{quadro}{chapter}
\renewcommand{\cftquadroname}{\quadroname\space} 
\renewcommand*{\cftquadroaftersnum}{\hfill--\hfill}

\setfloatlocations{quadro}{hbtp}

% O tamanho do parágrafo é dado por:
\setlength{\parindent}{1.3cm}

% Controle do espaçamento entre um parágrafo e outro:
\setlength{\parskip}{0.2cm}  % tente também \onelineskip

% ---
% compila o indice
% ---
\makeindex


\begin{document}

\selectlanguage{brazil}

% Retira espaço extra obsoleto entre as frases.
\frenchspacing 

\imprimircapa


\begin{dedicatoria}
   \vspace*{\fill}
   \centering
   \noindent
   \textit{ Este trabalho é dedicado às crianças adultas que,\\
   quando pequenas, sonharam em se tornar cientistas.} \vspace*{\fill}
\end{dedicatoria}

\begin{agradecimentos}

Escrever

\end{agradecimentos}

% ---
% RESUMOS
% ---

% resumo em português
\setlength{\absparsep}{18pt} % ajusta o espaçamento dos parágrafos do resumo
\begin{resumo}
Lalalala

 \textbf{Palavras-chave}: latex. abntex. editoração de texto.
\end{resumo}

% resumo em inglês
\begin{resumo}[Abstract]
 \begin{otherlanguage*}{english}
   This is the english abstract.

   \vspace{\onelineskip}
 
   \noindent 
   \textbf{Keywords}: latex. abntex. text editoration.
 \end{otherlanguage*}
\end{resumo}

% ---
% inserir lista de ilustrações
% ---
\pdfbookmark[0]{\listfigurename}{lof}
\listoffigures*
\cleardoublepage
% ---

% ---
% inserir lista de quadros
% ---
\pdfbookmark[0]{\listofquadrosname}{loq}
\listofquadros*
\cleardoublepage
% ---

% ---
% inserir lista de tabelas
% ---
\pdfbookmark[0]{\listtablename}{lot}
\listoftables*
\cleardoublepage
% ---

% ---
% inserir lista de abreviaturas e siglas
% ---
\begin{siglas}
  \item[Modelo] Fazer
\end{siglas}
% ---

% ---
% inserir lista de símbolos
% ---
\begin{simbolos}
  \item[$ \Gamma $] Letra grega Gama
\end{simbolos}
% ---

% ---
% inserir o sumario
% ---
\pdfbookmark[0]{\contentsname}{toc}
\tableofcontents*
\cleardoublepage
% ---



% ----------------------------------------------------------
% ELEMENTOS TEXTUAIS
% ----------------------------------------------------------
\textual

% \begin{figure}[htb]
% 	\caption{\label{fig_grafico}Gráfico produzido em Excel e salvo como PDF}
% 	\begin{center}
% 	    \includegraphics[scale=0.5]{imagens/Rackham_elves.jpg}
% 	\end{center}
% 	\legend{Fonte: \citeonline[p. 24]{araujo2012}}
% \end{figure}

\chapter{Visão geral}
\include{visaoGeral.tex}



\chapter{Gameplay e Macânicas}

Lorem ipsul dolort sit amet consetur

\section{\textit{Gameplay}}

Lorem ipsul dolort sit amet consetur

\section{Progressão do Jogo}

Lorem ipsul dolort sit amet consetur

\section{Estrutura de Missões/Desafios}

Lorem ipsul dolort sit amet consetur

\section{Objetivos}

Lorem ipsul dolort sit amet consetur

\section{Mecânicas}

Lorem ipsul dolort sit amet consetur

\section{Movimentação / Física}

Lorem ipsul dolort sit amet consetur

\section{Objetos}

Lorem ipsul dolort sit amet consetur

\section{Ações}

Lorem ipsul dolort sit amet consetur

\section{Combate}

Lorem ipsul dolort sit amet consetur

\section{Economia}

Lorem ipsul dolort sit amet consetur

\section{Movimentação / Física}

Lorem ipsul dolort sit amet consetur

\section{Opções do Jogo}

Lorem ipsul dolort sit amet consetur

\section{Salvar \& Replay}

Lorem ipsul dolort sit amet consetur

\section{Easter Eggs, Cheats, bônus}

Lorem ipsul dolort sit amet consetur


\chapter{Arte do Game}

\section{Elementos Visuais}

Lorem ipsul dolort sit amet consetur

\section{Elementos Sonoros}

Lorem ipsul dolort sit amet consetur

\chapter{Narrativa, Ambientação e Personagens}

\section{História e Narrativa}

Lorem ipsul dolort sit amet consetur

\section{Mundo do Jogo}

Lorem ipsul dolort sit amet consetur


\section{Visão Geral}

Lorem ipsul dolort sit amet consetur

\section{Áreas do Jogo}

Lorem ipsul dolort sit amet consetur


\section{Personagens}

\subsection{Ogof}
\subsection{Duende}
\subsection{Bruxa}
\subsection{Xamã}

Lorem ipsul dolort sit amet consetur

\section{Fases}

Lorem ipsul dolort sit amet consetur


\section{Fase de Treino e/ou Tutorial}

Lorem ipsul dolort sit amet consetur

\chapter{Interface}

\section{Sistema Visual}

Lorem ipsul dolort sit amet consetur


\section{Sistema de Controle}

Lorem ipsul dolort sit amet consetur

\section{Sistema de Ajuda}

Lorem ipsul dolort sit amet consetur

\chapter{Inteligência Artificial (AI)}


\section{Oponentes e AI Inimiga}

Lorem ipsul dolort sit amet consetur

\section{AI parceira ou não-inimigas}

Lorem ipsul dolort sit amet consetur

\section{AI de Suporte}

\chapter{Aspectos Técnicos}

\section{Plataforma de Produção}

\cite{shoemaker}
\cite{encyclopedia}
Lorem ipsul dolort sit amet consetur

\section{Hardware e Software de Desenvolvimento}

Lorem ipsul dolort sit amet consetur

\section{Requerimentos de Rede}

\chapter{Modelo de Negócio} 


\bibliography{00_bibliografia}
\end{document}
