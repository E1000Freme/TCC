\section{Nome do Jogo}
Ogof e o templo das joias.

\section{\textit{High Concept} do Jogo}
Conceito em 150 linhas


\section{Gênero}

Descreva e justifique o genero do jogo
Fantasia, puzzle

\section{Púbico Alvo}

Descreva e justifique o publico alvo do jogo
Adolescentes a partir de 16 anos....

\section{\textit{Game Flow}}

Ilustração ou tabela que demonstre todas as telas que o jogo terá e como se relacionam entre si

\section{Estilo estético}

Resumo

\section{Inspirações}
Apresente nesta seção jogos que foram utilizados pela equipe como inspiração na elaboração da ideia e/ou desenvolvimento do jogo.
Mostrar quais foram suas inspiraçÕes poderá ajudar o leitor a entender mais a respeito de seu jogo

\subsection{Outlander}
\subsection{Filha de Feiticeira}
\subsection{Desencanto}
\subsection{Os Vingadores - Guerra ininita}
\subsection{Franquia God of War}
\subsection{Sonhos de uma noite de verão}
\subsection{Brownies}
Brownies são 
\subsection{Alux}

\section{Equipe de Desenvolvimento}
É importante destacar que participação cada integrante da equipe tem ou teve no desenvolvimento do projeto.
Caso haja itens do projeto que não venham a ser desenvolvidos pelos integrantes da equipe, como por exemplo assets de áudio ou imagem que sejam baixados de bancos gratuitos, é importante relatar essa situação aqui para que se tenha uma noção exata da autoria dos vários componentes do projeto.

A equipe dividiu-se de forma a otimizar as forças de cada um dos integrantes, desta forma ficamos com a seguinte divisão de tarefas não rigidas. 

Marina atuou fortemente com o game design, character design e história. 

Pedro atuou como desenvolvedor e 

Eric atuou como arte técnica, animação e desenvolvimento