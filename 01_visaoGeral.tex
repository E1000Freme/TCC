\section{Nome do Jogo}
Colocar duas referencias
Ogof e o templo das joias.

\section{\textit{High Concept} do Jogo}
Ogof e o templo das joias é um game de fantasia, em que o jogador assumirá o papel de 3 diferentes figuras, e trabalhar em conjunto para progredir.
% Ogof e o templo das joias é um game de fantasia, exploração e puzzle em que você irá assumir o papel de 3 figuras de diferentes épocas do tempo, 
% % cada qual com suas forças e fraquezas, para progredir será necessário explorar as forças de cada personagem
% e trabalhar em conjunto para descobrir os planos de Ogof.

% Conceito em 150 caracteres


\section{Gênero}

O jogo será 3D em terceira pessoa, com fases que irão misturar diversas mecânicas como \textbf{puzzle, hack em slash} e exploração. Cada fase será linear com a dificuldade gradualmente aumentada até o final da fase, em que cada uma terá um miniboss.

% Descreva e justifique o genero do jogo

\section{Púbico Alvo}

Descreva e justifique o publico alvo do jogo
Adolescentes a partir de 16 anos....
\vfill
\pagebreak

\section{\textit{Game Flow}}
\begin{figure}[htb]
	\caption{\label{fig_grafico}Fluxo de telas}
	\begin{center}
	    \includegraphics[scale=0.5]{imagens/Flow.png}
	\end{center}
	\legend{Fonte: Própria Autoria}
\end{figure}
Ilustração ou tabela que demonstre todas as telas que o jogo terá e como se relacionam entre si

\section{Estilo estético}

Resumo
Colocar fontes, bibliografia colocar capa do livro/imagem referencia

\section{Inspirações}
\subsection{Outlander}
Sinopse: 
A trama inspirou todo o cenário da bruxa, escolha da cidade e relação com os vilões.
\subsection{Filha de Feiticeira}
Sinopse:

Uma das inspirações para a construção de personalidade e cultura da Bruxa.
\subsection{Desencanto}
Sinopse:

Uma das inspirações para a construção de todo o personagem Elfo.
\subsection{Os Vingadores - Guerra infinita}
Inspirou a proposta de coleta de joias e a personalidade de Ogof teve referência no personagem Thanos.
\subsection{Franquia God of War}
Sinopse:

Inspirou a animação, técnica de câmera e combate do gameplay.
\subsection{Sonhos de uma noite de verão}
Sinopse:

\subsection{Brownies}
Brownies são pequenos seres noturnos do folclore britânico, que cuidam dos afazeres da casa, se ofendem facilmente e evitam serem vistos pelos humanos \cite{britannica_2011}, traços de personalidades que utilizamos como inspiração para a formulação do Duende.
Todo: Inserir as referências na bibliografia, estou tendo dificuldades para achar os livros no formato digital.
\begin{figure}[htb]
	\caption{\label{fig_grafico}Brownie por Arthur Rackham }
	\begin{center}
	    \includegraphics[scale=0.5]{imagens/brownie.jpg}
	\end{center}
	\legend{Fonte: \citeonline{carolyn_2016}}
\end{figure}
\vfill
\pagebreak


\subsection{Alux}
Alux são pequenas criaturas da mitologia de alguns povos Mayas, costumeiramente são invisíveis, mas podem se tornar visíveis para se comunicar e assutar humanos.  São responsáveis pela boa colheita, espantando animais e controlando a chuva \cite{judith_2009}. 
\begin{figure}[htb]
	\caption{\label{fig_grafico}Alux}
	\begin{center}
	    \includegraphics[scale=0.5]{imagens/alux.jpg}
	\end{center}
	\legend{Fonte: \citeonline{judith_2009}}
\end{figure}
\vfill
\pagebreak

\subsection{Siempre Bruja}
A série que conta a história de uma bruxa negra e escrava que é transportada  pelo tempo para enfrentar um grande bruxo inspirou os poderes da bruxa e itens com os quais trabalha. 

\section{Equipe de Desenvolvimento}
Tabelar
A equipe dividiu-se de forma a otimizar as forças de cada um dos integrantes, desta forma ficamos com a seguinte divisão de tarefas não rígidas. 

Marina atuou fortemente com o game design, character design e história. 

Pedro atuou como programador e game design

Eric atuou como arte técnica, animação e programação