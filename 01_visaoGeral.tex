\section{Nome do Jogo}
Ogof e o templo das joias.

\section{\textit{High Concept} do Jogo}
Ogof e o templo das joias é um game de fantasia, em que você assumirá o papel de 3 diferentes figuras, e trabalhar em conjunto para progredir.
% Ogof e o templo das joias é um game de fantasia, exploração e puzzle em que você irá assumir o papel de 3 figuras de diferentes épocas do tempo, 
% % cada qual com suas forças e fraquezas, para progredir será necessário explorar as forças de cada personagem
% e trabalhar em conjunto para descobrir os planos de Ogof.

% Conceito em 150 caracteres


\section{Gênero}

O jogo terá elementos de fantasia, exploração e puzzle....

% Descreva e justifique o genero do jogo

\section{Púbico Alvo}

Descreva e justifique o publico alvo do jogo
Adolescentes a partir de 16 anos....

\section{\textit{Game Flow}}

Ilustração ou tabela que demonstre todas as telas que o jogo terá e como se relacionam entre si

\section{Estilo estético}

Resumo

\section{Inspirações}
Apresente nesta seção jogos que foram utilizados pela equipe como inspiração na elaboração da ideia e/ou desenvolvimento do jogo.
Mostrar quais foram suas inspiraçÕes poderá ajudar o leitor a entender mais a respeito de seu jogo

\subsection{Outlander}
\subsection{Filha de Feiticeira}
\subsection{Desencanto}
\subsection{Os Vingadores - Guerra ininita}
\subsection{Franquia God of War}
\subsection{Sonhos de uma noite de verão}
\subsection{Brownies}
Brownies são pequenos seres noturnos do folclore britânico, que cuidam dos afazeres da casa, se ofendem facilmente e evitam serem vistos pelos humanos, traços de personalidades que utilizamos como inspiração para a formulação do Duende.
Todo: Inserir as referências na bibliografia, estou tendo dificuldades para achar os livros no formato digital.

\subsection{Alux}
Alux são pequenas criaturas da mitologia de alguns povos Mayas, costumeiramente são invisíveis, mas podem se tornar visíveis para se comunicar e assutar humanos.  São responsáveis pela boa colheita, espantando animais e controlando a chuva. 

\section{Equipe de Desenvolvimento}
É importante destacar que participação cada integrante da equipe tem ou teve no desenvolvimento do projeto.
Caso haja itens do projeto que não venham a ser desenvolvidos pelos integrantes da equipe, como por exemplo assets de áudio ou imagem que sejam baixados de bancos gratuitos, é importante relatar essa situação aqui para que se tenha uma noção exata da autoria dos vários componentes do projeto.

A equipe dividiu-se de forma a otimizar as forças de cada um dos integrantes, desta forma ficamos com a seguinte divisão de tarefas não rigidas. 

Marina atuou fortemente com o game design, character design e história. 

Pedro atuou como programador e 

Eric atuou como arte técnica, animação e programação