\section{\textit{Gameplay}}

Durante as três primeiras fases, o jogador é apresentado a um dos personagens em seu ambiente natural, este é o momento para que se aprenda as diversas habilidades dos diferentes personagens.

As habilidade serão apresentadas ao jogador gradualmente, sem que o mesmo fique sobrecarregado de informações, ao mesmo passo que a dificuldade de inimigos também aumentará, objetivando manter o \textit{flow} do jogo.

Ao final das duas primeiras fases, o jogador iniciara uma nova fase com um novo personagem, já no  final da terceira, ele será levado também a uma nova fase, porém agora ele deverá controlar os três personagens de maneira cooperativa, para que unindo suas habilidades únicas progrida no jogo.




\section{Progressão do Jogo}

TODO: Descreva como o jogo evolui e o que o jogador precisa fazer para progredir no jogo.

\section{Estrutura de Missões/Desafios}

TODO: Descreva os desafios micros e macros dentro de cada fase do jogo.

\section{Objetivos}

Nas três primeiras fases, os personagens tem por objetivo, achar aquilo que lhes foi roubado, através de dicas que lhes serão dadas pelo caminho.

Na ultima fase, devem sobreviver e utilizar habilidade me conjunto para parar o antagonista, e assim recuperar o que lhes foi tomado.

\section{Mecânicas}

TODO: Descreva as regas do jogo (implícitas e explícitas) e como as partes (personagens do jogo) interagem entre si.

Todos os personagens poderão andar livremente pelo cenário, contudo existem mecânicas específicas de cada um dos personagens que estão detalhadas abaixo:


\subsection{Duende}
\subsubsection{Crescer plantas}
Segue os mesmos princípios de coleta das maçãs. Quando acionadas causam chuvas que podem ser ácidas, para afastar inimigos, ou férteis fazendo crescer árvores e vegetações usadas em beneficio do jogador, tais como pontes, troncos onde ele pode se esconder, etc.
\subsubsection{Destruir plantas}
\subsubsection{Invisibilidade}
Quando uma maçã é coletada ela pode ficar guardada para um momento de conveniência. Quando acionada fica invisível por 5 segundos.

\subsection{Xamã}

\subsubsection{Curar}
\subsubsection{Olhos de Águia}

\subsection{Bruxa}


\section{Movimentação / Física}
A movimentação dentro do jogo se dará utilizando as teclas W, A, S, e D do teclado, por ser um padrão já conhecido e utilizado em muitos jogos.

Com o mouse, o \textit{player} conseguirá rotacionar a camera do jogo, para ter uma melhor visibilidade do cenário, e assim maiores chances de resolver os desafios.

Será utilizado o motor de física fornecido pela própria Unity3D,

Para maiores detalhes consulte a sessão 5.2 - Sistema de Controle deste documento.

\section{Objetos}

TODO:

\section{Ações}

Descreva como o jogador interage de forma geral (controles, botões, objetos) com o mundo do jogo.

\subsection{Xamã}
\begin{itemize}
\item Caça usando arco e flecha, um instrumento mais avançado em relação à rudimentar lança
\item Construção de moradias em cavernas, pés e paredes de desfiladeiros
\item Construção de Kivas, uma sala ampla dentro das cavernas, utilizada principalmente para discussões de teor político e rituais religiosos, mas que também vinha a servir como ponto de encontro; transmissão oral de conhecimentos, sem que muito registro histórico escrito tenha sido deixado para maiores referencias;
\item Substituição da caça por plantações agrícolas, possibilitando a alimentação de mais aldeões.
\item Habilidade para criar equipamentos de caça e sobrevivência
\end{itemize}

\section{Combate}

Descreva como o jogador interage de forma geral (controles, botões, objetos) com o mundo do jogo.

\section{Economia}

Não há economia planejada para o jogo.

\section{Opções do Jogo}

O jogo contará com variados níveis de dificuldades, que afetarão como a IA inimiga irá se comportar, bem como pontos de vida e ataque

Será possível também ativar e desativar legendas para as partes de histórias e diálogos no jogo, e por fim o jogador poderá selecionar a resolução do jogo que prefere jogar.

\section{Salvar \& Replay}

O jogo contém um sistema de auto-salvamento, que será ativado por certas partes chaves do jogo. Além disso será possível salvar pelo menu de pause do jogo, para retornar futuramente no estado em que parou.

\section{Easter Eggs, Cheats, bônus}

No momento não estão sendo planejados Easter Eggs ou Cheats, contudo durante o jogo alguns objetos coletáveis darão acesso às artes conceituais do jogo.
