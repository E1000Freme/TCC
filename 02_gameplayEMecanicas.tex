\section{\textit{Gameplay}}

Durante as três primeiras fases, o jogador é apresentado a um dos personagens em seu ambiente natural, este é o momento para que se aprenda as diversas habilidades dos diferentes personagens.

As habilidade serão apresentadas ao jogador gradualmente, sem que o mesmo fique sobrecarregado de informações, ao mesmo passo que a dificuldade de inimigos também aumentará, objetivando manter o \textit{flow} do jogo.
\begin{citacao}
Na procura de explicar o que torna uma experiência divertida, Mihaly Csikszentmihaly entrevistou, observou e analisou milhares de dados de pessoas que se dedicavam a diversas atividades, desde jogos de xadrês a montanhismo.

[\ldots]

O resultado deste estudo originou a teoria do \textit{flow}. No estado de \textit{flow} uma pessoa está a viver uma experiência ótima que é tao gratificante, que leva a pessoa a estar imersa na atividade, sem se preocupar com o que retira e experiencia ou com o facto de esta ser potencialmente perigosa. Este é um estado de motivação intrínseca extremamente focada. 

Nesse estado pode dizer-se que a pessoa está extremamente engajada com a experiência e seu nível de interesse/atenção e empenho/motivação é elevado. 

[\ldots]

Para além destes elementos, foi identificada a necessidade de haver equilíbrio entre a dificuldade da tarefa e a perícia da pessoa. Se em algum caso, este equilíbrio se perder, o estado de \textit{flow} também se perde. \cite{pradadesign}
\end{citacao}

Ao final das duas primeiras fases, o jogador iniciara novo nível com um novo personagem, ao final da terceira, ocorrerá o encontro dos três personagens em mais um ambiente, o jogador então deverá controlar os três personagens de maneira cooperativa, para que unindo suas habilidades únicas progrida no jogo.


\section{Progressão do Jogo}
O jogo inicia-se com o Duende, após presenciar o sequestro de sua querida esposa, próximo encontra-se o chefa da vila, que irá orientá-lo a buscar pelo sábio que vive no topo do fiorde, O controle então é passado para o jogador que deverá seguir o caminho até o velho sábio, pelo caminho deverá enfrentar inimigos e outras criaturas da floresta. Apos dialogar com o sábio, é lhe dada a missão de coletar alguns itens mundanos, que juntos farão parte de um ritual para que e Duende consiga ir atrás de sua amada, feito o ritual um portal irá se abrir e dever-a ser atravessado.

Assim que o Duende atravessar o portal um outro nível se iniciará agora com o personagem do Xamã, este receberá uma mensagem de alguns índios de sua tribo, ele então deverá procurar um lugar mais alto procurando visualizar as construções de seu povo contudo o que consegue ver está muito longe, percebe então no céu o cantar de uma águia a observa por um momento e então o animal voa ao encontro do personagem e pousa em seu braço, após uma trocar de olhares ele lança-a de volta a voo e então o jogador estará controlando a águia que viaja muito mais veloz que o andar do personagem, deverá então buscar por uma pessoa esguia puxando outra aparentemente desmaiada, ao observar isso, o jogador perde o controle da águia, e volta a controlar o Xamã, que irá se encaminhar a sua vila a procura de alguns itens, coletados ele deverá voltar ao local inicial, para reabrir o portal.

Atravessado o portal o jogador estará em uma clareira e dentro da floresta um lampejo de luz será observado, agora ele estará controlando a personagem Bruxa, que deverá se encaminhar em direção e este lampejo, ao chegar em um riacho uma coleira lhe chama a atenção, ao pegar essa coleira uma nova missão iniciara, e ela deverá ir até a cidade para completa-la, ao final desta fase, a personagem será transportada por um portal.

Após a Bruxa atravessar o portal, os três personagens irão se encontrar em uma sala piramidal, com alguns desenhos de portas, olhar-se-ão confusos e com um rápido dialogo todos percebem que estão ali pelo mesmo motivo, o controle é então devolvido ao jogador, que aprenderá a habilidade de trocar de personagem por desenhos nas paredes. Assim que trocar de personagens algumas vezes um portal irá se abrir em uma das portas, ele deverão entrar neste portal.

Passado o portal anterior ele encontram-se na base do templo das Joias, e um tufão se forma no topo de uma das piramides, deverão apressar-se para evitar que o antagonista conclua seu ritual.


\section{Estrutura de Missões/Desafios}

Como o jogo é um \textit{hack'n'slash - puzzle}, vide 2.1 \textit{Gameplay} o jogador estará sempre sendo desafiados por diferentes inimigos, deverá utilizar do raciocínio lógico em conjunto com as habilidades já aprendidas para transpor os elementos do cenário, e também será necessário coletar objetos para diferentes ações durante o jogo, alguns destes objetos serão escassos cabendo ao jogador se planejar para o melhor momento de consumi-los.

\section{Objetivos}

Segundo Jeremy Gibson, em seu livro \textit{"Introduction to Game Design, Prototyping and Development"}, por mais que todo jogo contenha um objetivo simples, finalizar e/ou ganhar o jogo, o jogador está a todo momento analisando diversos objetivos durante o processo. Gibson ainda divide esses objetivos em curto, médio e longo prazo. \cite{gibson2014}

Em Ogof e O Templo das Joias os objetivos de curto prazo são sobreviver aos diferentes combates e desafios para avançar no jogo. Como objetivo de médio prazo, o jogador não quer perder o rastro deixado pelo antagonista, e assim achar quem lhe foi tirado. Por fim, o objetivo de longo prazo é impedir que Ogof cumpra o ritual.

\section{Mecânicas}

Os personagens podem movimentar-se livremente pelo cenário, como descrito na sessão 2.6 Movimentação / Física. Além disso os personagens têm habilidades específicas, e como são de ambientes e tempos diferentes, percebem o cenário a sua volta de forma única. Isto é, alguns elementos podem ser vistos apenas por um dos personagens, e serão elementos chave para a progressão do jogo.

Abaixo está descrito, para cada um dos personagens controlados pelo jogador quais serão suas habilidades únicas. Estas terão cada uma sua barra de energia, e quando ativadas, consumirão desta energia.

\subsection{Duende}
\subsubsection{Crescer plantas}
Inspirado nas lendas dos Alux(Vide 1.6 Inspirações), que conseguem controlar os ventos e as chuvas, será possível crescer plantas em qualquer lugar, contanto que haja um substrato para a planta crescer, isto é,  madeira, solo ou água. Locais com pedras ou muito secos não permitirão que plantas se germinem.

Estas plantas serão utilizadas como plataformas para acessar áreas que precisem de uma plataforma ou ``elevador'' para serem acessadas.

Para ativar essa habilidade, utilizar-se-á o botão de habilidade 1 (vide 5.2 Sistema de Controle), e ao ativá-la será consumido sua ``energia de chuva''. Caso essa energia se esgote, não será possível ativar esta habilidade. Para recuperar esta energia ele deverá recolher fragmentos.


\subsubsection{Destruir plantas}
Esta habilidade deriva da habilidade anterior, mas irá destruir plantas existentes, ou mesmo plantas que foram criadas por "Crescer Plantas". Para ativá-la, deve-se utilizar o botão de habilidade 2 (Vide 5.2 Sistema de controle). Esta  habilidade consumirá do mesmo recurso que sua habilidade anterior. 


\subsubsection{Invisibilidade}
Esta habilidade deixará o personagem invisível para os inimigos por um certo tempo, assim como as habilidades anteriores, ela terá uma barra de energia, que será consumida enquanto estiver invisível. Para recuperar a energia de invisibilidade, deverá coletar fragmentos.

Esta habilidade está mapeada para o botão de habilidade 3 



\subsection{Xamã}
\subsubsection{Curar}
Inspirado nos curandeiros das culturas Nativo-Americanas, em especial nos Pueblo, e nas tribos que acredita-se que tenham ancestralidade Pueblo, esta habilidade irá curar uma porcentagem da vida do objeto alvo, seja ele aliado, inimigo ou outro ser com vida.

Para utilizar essa habilidade será utilizado o botão de habilidade 1

\subsubsection{Rastrear}
O personagem devera ter um objeto em mão para que possa utilizar esta habilidade, quando esta condição for satisfeita e utilizar esta habilidade com o botão de habilidade 2, o personagem conseguirá ver por um tempo determinado dicas no cenário relativas ao objeto.

\subsubsection{Olhos de águia}

O Xamã poderá evocar uma águia, para que possa visualizar dor outra perspectiva o ambiente em que se está. Ao ativar essa habilidade o jogador terá alguns segundos para controlar a águia, com uma mecânica de voo. Após esse tempo, a câmera do jogo voltará a focar no Xamã.

Além de voar, a Águia tem a vantagem de ser mais rápida que os personagens e não chamar a atenção de inimigos.

Esta habilidade não tem barra de energia a ser consumida, mas não poderá ser utilizada quando em combate, e será ativada com o botão de habilidade 3. 


\subsection{Bruxa}

A Bruxa também terá três habilidades, mas estas serão definidas futuramente no projeto. 

\subsection{Ataques}
Em conjunto com as habilidades, cada personagem terá dois ataques, um com o botão direito do mouse, ataque rápido porém pouco efetivo, e outro com o botão esquerdo do mouse, ataque mais forte contúdo deixara o personagem vulneravel por instantes.


\section{Movimentação / Física}
A movimentação dentro do jogo se dará utilizando as teclas W, A, S, e D do teclado, por ser um padrão já conhecido e utilizado em muitos jogos. Para maiores detalhes consulte a sessão 5.2 - Sistema de Controle deste documento.
O projeto está sendo feito com o motor de jogo\footnote{Do inglês \textit{Game Engine}} Unity3D, que já fornece uma solução completa e customizável para a física do jogo, optou-se por utiliza-lá sem grandes alterações.


\section{Objetos}

\subsection{Pedaço de roupa da esposa do Duende}
Um pedaço de roupa que será utilizado para definir o destino correto do portal quando o Duende o fizer.

\subsection{Pedaço de roupa do filho do Xamã}
Um pedaço de roupa que o Xamã utiliza-rá para rastrear ser filho.

\subsection{Coleira do gato da Bruxa} 
Este lembrete do companheiro de nossa personagem tem o poder de ajuda-la a localiza-lo pela conexão espiritual.


\section{Ações}
O jogador poderá se movimentar livremente pelo cenário, bem como atacar, pular, interagir, coletar e utilizar as habilidades específicas de cada um dos personagens.

Para maiores detalhes do uso de tais ações, verifique a sessão 5.2 Sistemas de Controle.

\section{Combate}
O combate irá ganhar dificuldade de forma gradual enquanto o jogador avança pelo jogo, com diferentes criaturas de diferentes níveis de poder e tipos de ataque (corpo a corpo e à distancia, por exemplo), e pontos de vida. 

Para manter a sensação de urgência ao jogador e fornecer um melhor sistema de combate, utiliza-se um comportamento de inteligência artificial conhecido como círculo de batalha, presente em diversos jogos como God of War, Dark Souls, Okami e outros \cite{BattleCi}.

Para maiores detalhes leia a sessão 6.1 - Oponentes e IA Inimiga

\section{Economia}

Não há economia planejada para o jogo.

\section{Opções do Jogo}

Segundo o livro Design e Desenvolvimento de Jogos, uma das formas de se identificar jogadores é classificá-los como \textit{hardcore}, que são jogadores experientes que gostam de ser desafiados, e casuais, que são jogadores que preferem jogos mais fáceis e costumeiramente abandonam jogos que são demasiadamente desafiadores \cite{pradadesign}. 

O livro ainda indica que para atingir ambos os públicos é interessante que se tenha ao menos dois níveis de dificuldades em um jogo. Assim, Ogof e O Templo das Joias contará com variados níveis de dificuldade, que afetarão como a IA inimiga irá se comportar, bem como pontos de vida e ataque.

Será possível também ativar e desativar legendas para as partes de histórias e diálogos no jogo. E por fim o jogador poderá selecionar a resolução do jogo que prefere jogar.

\section{Salvar \& Replay}

O jogo contém um sistema de auto-salvamento, que será ativado em certos momentos chave do jogo. Além disso será possível salvar pelo menu de pause do jogo, para retornar futuramente no estado em que parou.

\section{Easter Eggs, Cheats, bônus}

No momento não estão sendo planejados Easter Eggs ou Cheats, contudo durante o jogo alguns objetos coletáveis darão acesso às artes conceituais do jogo.
