\section{\textit{Gameplay}}
\begin{enumerate}
\item Duende
\end{enumerate}
\begin{itemize}
\item Maçã:
\end{itemize}
Quando uma maçã é coletada ela pode ficar guardada para um momento de conveniência. Quando acionadende fica invisivel por 5 segundos.
\begin{itemize}
\item Banana:
\end{itemize}
Segue os mesmos princípios de coleta das maçãs. Quando acionadas causam chuvas que podem ser ácidas, para afastar inimigos, ou férteis fazendo crescer árvores e vegetações usadas em beneficio do jogador, tais como pontes, troncos onde ele pode se esconder, etc.

\section{Progressão do Jogo}

Descreva como o jogo evolui e o que o jogador precisa fazer para progredir no jogo.

\section{Estrutura de Missões/Desafios}

Descreva os desafios micros e macros dentro de cada fase do jogo.

\section{Objetivos}

Descreva os desafios micros e macros dentro de cada fase do jogo.

\section{Mecânicas}

Descreva as regas do jogo (implícitas e explícitas) e como as partes (personagens do jogo) interagem entre si.

\section{Movimentação / Física}

Descreva como funcionará as movimentações dos personagens dentro jogo (as limitações e possibilidades) e como será o sistema de física.

\section{Objetos}

Descreva os objetos que existem no jogo e como o jogador interage com eles.

\section{Ações}

Descreva como o jogador interage de forma geral (controles, botões, objetos) com o mundo do jogo.

\section{Combate}

Descreva como o jogador interage de forma geral (controles, botões, objetos) com o mundo do jogo.

\section{Economia}

Descreva como funciona o sistema financeiro (moedas) e como elas são utilizadas no mundo do jogo.

\section{Opções do Jogo}

Descreva quais são as modalidades de jogo e como elas afetam/alteram o jogo.

\section{Salvar \& Replay}

Descreva o funcionamento do sistema de save ou autosave e se há modo New Game Plus.

\section{Easter Eggs, Cheats, bônus}

Descreva se o jogo possuir conteúdos escondidos, trapaças ou conteúdos extras (roupas, modos, etc.).