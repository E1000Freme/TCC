
\section{História e Narrativa}

O mago Ogof, sedento por poder, capturou três seres místicos, em três diferentes épocas para o seu ritual ardiloso. Quando os planetas se alinham ele captura o gato de uma bruxa, a esposa de um elfo e o filho de um xamã. O ritual espalhou pelo mundo 3 joias que guardam as almas dos raptados. Resgate as joias com esses três personagens jogáveis, mas o ritual só é reversível com as três joias juntas.

\section{Áreas do Jogo}

\subsection{O mundo de Ogof}
Mesopotâmia, idade antiga. Ogof vive na cidade de Ur na época de 4000 a.C. à 1900 a.C. Uma sociedade hidráulica onde todo o estilo de vida gira em torno de seus rios Tigre e Eufrates. Para tal, desenvolveram grandes habilidades arquitetônicas para lidar com as enchentes .

A sociedade residente de Ur é estamental, isto é, pessoas nascidas em uma determinada classe social morrerão nessa mesma classe e sua política tem bases religiosas. Isso significa que um líder político é necessariamente alguém que se diz parente ou conhecido de algum deus.

\subsection{O mundo dos Duendes}
Baseada em Geiranger, Fulkominn se ergue em meio a um fiorde das terras norueguesas, isolada do mundo montanhas rochosas de topos nevados é lar de uma pequena e tranquila sociedade de duendes. Para chegar até esta vila é preciso muita bravura e habilidade, as silhueta sinuosa dos fiordes dificulta muito a navegação, e ainda assim a forma mais fácil de se chegar é por mar. 

Para os pequenos moradores de tal cidade essa é inclusive uma tarefa corriqueira, e também conhecem passagens secretas que apenas seu tamanho diminuto permitem utilizar, isso permite que transitem com facilidade pelo mundo dos humanos, e caso julguem estes como de boa índole e dignos, ajudam estes humanos em diversas tarefas, seja cuidando de suas plantações\footnote{\citeonline{storniolo2009out}}, ou finalizando trabalhos manuais\footnote{\citeonline{grimm2003elves}}, entretanto se percebem má índole, podem atrapalhar e fazer o contrário do já dito, escondendo coisas, assustando animais e fazendo ruídos assustadores\footnote{\citeonline{carolyn_2016}}.

Não se sabe ao certo, mas é lei, não poder ser visto por humanos. Em algumas histórias como "O Conto do Sapateiro" narra-se que humanos viram duendes lhes ajudando, e cheios de gratidão lhes presentearam com roupas\footnote{\citeonline{grimm2003elves}}, contudo receber um presente de um humano é a prova cabal de que sabem de sua existência, e pela lei, o duende presenteado é banido de Fulkominn, e deixado a própria sorte.

\subsection{O mundo da Bruxa}
Aqui, teremos um mundo baseado na Vila de Killin, como dito na sessão 1.6, mas o contexto em que se insere será desenvolvido futuramente.


\subsection{O mundo do Xamã}
O Mundo do Xamã se dá em volta da história dos Pueblos, uma tribo indígena norte-americana. Mais informações na seção 1.7.H.

Aqui, toda a população vive ao pé de Cânions, levantando construções em túneis pra dentro deles. As moradias são pequenas, mas são de fácil acesso por estarem muito próximas umas as outras. A estrutura da povoação abriga casas, mercados e centros de reuniões gerais de interesse geral, como eventos de cunho religioso, cerimônias e discussão de desavenças entre moradores. Fora do Cânion, há plantações que fornecem comida para os moradores, sendo essa uma das muitas fontes de alimentação. Há também a caça, que usa artefatos como lanças e flechas, junto com as mercadorias vendidas no comércio local.

Em seu meio, o Xamã é um conselheiro experiente, que guia as pessoas nos caminhos a seguir na vida.

\section{Personagens}
Para a melhor construção dos personagens procurou-se utilizar Arquétipos, que são tipos de personagens proposto pela teoria de Carl Jung, e são utilizados para aumentar a conexão entre público e o produto de entretenimento.

\subsection{Ogof}
Ogof, antagonista do jogo, é um mago ancião que em tempos áureos lutou com todas suas forças para que a terra mantivesse seu equilíbrio e paz, contudo por fatores não totalmente conhecidos renega tudo que fez e torna-se aquilo que jurou combater, sedento por conhecimento e poder passa sua vida em busca de algo que não se sabe o que é. Para maiores detalhes consulte o anexo A.

Procurou-se aplicar o arquétipo do Sombra, que costumeiramente é o inimigo de uma trama e um personagem que representa o lado sombrio da história.


% \textbf{Back history: }Sedento por conhecimento, acessou ensinamentos ocultos que desestabilizaram sua mente e agora o poder é sua obsessão.

% \textbf{Físico:} velho, parrudo, imponente. Com uma barba longa e grisalha.

% \textbf{Descrição psicológica:} Impiedoso e insano, ocasionalmente fala coisas sem sentido.

% \textbf{Arquétipo:} Sombra

\subsection{Duende}
Um pequeno e trabalhador Duende, não é feliz com o que precisa desempenhar durante seu dia a dia, irritadiço, sempre de mau humor e com respostas ácidas, a única coisa que lhe prende a sua vila é sua amada esposa, a quem protege e defende com todas suas forças.

% \textbf{Back history:} Trabalhador com uma rotina tediosa e repetitiva. Seu mal humor é apenas curado por sua esposa doce.

% \textbf{Descrição física:} Pequeno, gordo e mal vestido.

% \textbf{Descrição psicológica:} Temperamental, revoltado e mau humorado.

% \textbf{Arquétipo:} Pícaro

% \textbf{Temperamento:} Irritadiço quando está com o contador de vida cheio. Quando o contador está próximo do fim fica furioso.

% \textbf{Iteração: } Neste universo, duendes são seres misticos que coexistem com humanos. Parte dos desafios é manter-se despercebido. São mais facilmente vistos por crianças.

% \textbf{Habilidades: }
% \begin{itemize}
% \item  Invisibilidade: Auxilia na furtividade do personagem mas depende diretamente dos recursos que o jogador possui. Ao longo da fase deverá coletar maçãs que encherão uma barra de energia que, quando completa, libera a invisibilidade. O jogador deve gerenciar a barra para usá-la em obstáculos estratégicos.
% \end{itemize}


% \begin{itemize}
% \item Chuva:  funciona da mesma forma que a invisibilidade, ao coletar uma quantidade de bananas suficiente para encher a barra poderá invocar chuva, mas deverá administrar os recursos com cautela.
% \end{itemize}
% Ambos os itens, maçãs e bananas são adquiridos tanto buscando-os em esconderijos no cenário quanto prestando favores aos humanos.

% \textbf{Fraqueza:} De acordo com [N˜AO LEMBRO QUEM DISSE ISSO] duendes são seres desatentos e facilmente distraíveis. Para simular essa característica, o jogador só pode enxergar com clareza poucos metros de distância a partir do duende. Ao restante do cenário foi adicionado um desfoque para que seja levemente borrado, dificultando a identificação dos elementos.

\subsection{Bruxa}
Muito estudiosa, sua curiosidade a moveu a conhecer as artes do oculto até que encontrou nas velhas anciãs, mentoras para seus estudos, aprendeu a respeitar as artes mágicas e nunca faz uso sem autorização. 

Em sua juventude optou por isolar-se em uma clareira na floresta onde recebe muitas vísitas, mas seus principais companheiros são seu gato preto, e seus lívros.



% \textbf{Back history:} Estudiosa e conhecedora das artes do oculto. Respeita as artes mágicas e não faz uso sem autorização. Treinada pelas anciãs.

% \textbf{Descrição física/psicológica:} Metamorfa, sua personalidade é afetada pela energia do ambiente. Quando suas energias são boas e prestativas assume a aparência de bruxa má. Quando a escuridão a consome se torna uma mulher bela e jovem.

% \textbf{Arquétipo: }Camaleão

\subsection{Xamã}
O trabalho de um Xamã, não é nada fácil, por isso é escolhido na mais tenra juventude e então passa a ser treinado pelos mais velhos da vila, e o conhecimento do velho Xamã é despejado à este jovem de forma contínua ao longo dos anos, até o momento que na morte de seu mestre, ele estará apto a assumir seu posto.

Como foi escolhido muito jovem este personagem tem uma atitude bastante taciturna, e sempre procura atender e ajudar aqueles que lhe procuram e estão a sua volta.


% \textbf{Back history: }indígena norte americano, sábio e orientador da tribo.

% \textbf{Descrição física: }Pele vermelha, e aparência respeitável. Inspira confiança mas intimida. Utiliza pele de lobo na cabeça.

% \textbf{Descrição psicológica:} Sério e protetor como um grande pai. Transcende o corpo e trabalha com espíritos para proteger o grupo.

% \textbf{Arquétipo:} Mentor

\section{Fases}

\subsection{A Redenção de um duende -- Fullkomnun}
O duende chega a tempo de ver Ogof levando sua esposa e o persegue até conseguir arrancar-lhe um pedaço da roupa, mas não consegue impedi-lo. O mestre da vila se aproxima e encontra o duende enfurecido. Percebendo que ele tem uma pedaço de roupa nas mãos,  manda que encontre o ancião a procura de ajuda.
O ancião então lhe diz que, para abrir o portal pelo qual Ogof fugiu serão necessários itens humanos [A DEFINIR] e itens de agricultura. O duende então sai em busca destes itens.

\subsubsection{O pico do fiorde}
O  ancião vive no ponto mais alto do reino: o topo de um icônico fiorde. Sob o ar rarefeito e a densidade da nevoa, apenas os duendes mais resistentes e determinados conseguem chegar a sua morada e assim receber como recompensa as suas sábias orientações.

\subsubsection{A plantação de Elvdans}
A cidade de Elvdans é povoada por camponeses humanos, um povo simples e supersticioso, porém amigável. O jogador deverá encontrar uma plantação suntuosa na cidade, onde uma família de duendes deixara itens mágicos para trazer prosperidade. Os fazendeiros não necessitam mais delas, você deve levá-las consigo para atender a ordem do ancião e posteriormente deixa-los a vista para que outro humano os pegue. Duendes são uma especie temperamental e emotiva por isso tem tendências a ajudar pessoas ao longo de suas viagens. Quando o usuário interromper seus afazeres para auxiliar alguém receberá itens que lhe serão uteis mais tarde.

\subsection{O Lobo Guardião}

O Xamã meditava concentrado quando um índio interrompe seu momento mais intimo para avisar que algo acontecera com seu filho. O protagonista corre para a sua casa e a encontra revirada e vazia.

No topo do cânion onde vivem os aldeões, existe uma águia mística que serve ao Xamã. Ela o ajudará a encontrar Ogof.

\subsubsection{O topo do cânion}
Saindo da casa do Xamã, o jogador deve seguir por tuneis obscuros para chegar ao topo do cânion onde encontrará a águia. Conectando-se com o animal o personagem será capaz de ver através de seus olhos. A ave sobrevoa os arredores mostrando rastros do caminho de Ogof até uma passagem onde um portal fora aberto.

\subsubsection{O Mercado}
Com as informações concedidas pela águia, o Xamã deve agora voltar pelos túneis e seguir até o mercado local que consiste em uma pequena feira ao ar livre de onde pode-se comprar os itens necessários para reabrir o portal.

O personagem deve coletar itens ao longo do caminho para forjar um arco e flecha que servirá como moeda de troca para a mercadoria.

\subsubsection{Portal}
Seguindo todos os rastros de Ogof, o Xamã chega até o outro lado do cânion. Adentra uma passagem e alinha os itens que comprou no mercado em cima de um pedestal, uma passagem outrora oculta se abre.

\section{Fase de Treino e/ou Tutorial}

O treino de habilidades será feito de forma imersiva para o jogador, nas fases em que ele jogará com apenas um personagem, sendo elas apresentadas de forma gradual ao jogador.
