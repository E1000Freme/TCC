
\section{Plataforma de Produção}

Por ter uma maior permeabilidade do mercado, optou-se em produzir o jogo principalmente para PC, por ter as teclas do teclado e mouse para interação, o que permite uma grande combinação de interações dentro do jogo. Além disso é muito fácil o jogador remapear as teclas de um teclado para ficar mais confortável para si.

Além do já citado no paragrafo anterior, produzir para pc nos permite trabalhar com maior detalhes e/ou mapas maiores pois esta plataforma tem um poder de processamento maior que plataformas mobile.

% O jogo será produzido para PC, por ter maiores opções de interações assim como um maior processamento, nos permitindo maior permeabilidade com o 3D.
% Descreva e justifique para quais plataformas o jogo está sendo produzido (PC, Android, consoles etc).


\section{Hardware e Software de Desenvolvimento}

O jogo será produzido em Unity3D devido ao time já ter pré conhecimento na ferramenta e na linguagem de programação(C\#) utilizada por ela. Além disso, tem sido um dos motores que mais crescem tanto no mercado nacional como internacional.

Para a modelagem será utilizado Maya e Blender, por familiaridade do time.

Para o desenvolvimento de imagens de conceito, será utilizado Photoshop, Illustrator, Krita, e Gravit Designer, os dois últimos programas são opções gratuitas aos dois primeiros sendo uma boa opção para gráficos \textit{bitmap} e  vetorial respectivamente.

O código do jogo será distribuído entre os desenvolvedores através da ferramenta de versionamento Git e também a ferramenta Git-lfs por se tratar de um projeto grande, em um repositório remoto no GitLab. Além disso, por conta o tamanho e quantidade de arquivos, a seção de objetos do jogo será também distribuída numa ferramenta a parte, armazenado em nuvem.

Como o projeto estará hospedado na plataforma GitLab, utilizaremos a área de organização de projetos, e rastreamento de problemas, fornecidos pela plataforma para gerenciar o andamento do projeto.


\section{Testes}

Na sessão 8.2 próximos passos é declarado que serão feitos testes quinzenais com diferentes jogadores para progressão coesa do projeto e garantir que atinja-se os objetivos citados durante este documento. 

\subsection{Software}

Os testes de software serão feitos utilizando as próprias soluções fornecidas pela \textit{Unity3D}, que pode ser encontrada na documentação da ferramenta sob o termo \textit{UnityTestRunner}

\subsection{Usabilidade}

Para os testes de usabilidade serão selecionados jogadores de testes, e que poderemos utilizar a ferramenta da Unity3D, \textit{remote settings} que nos permite atualizar os projetos desses jogadores, bem como acompanhar pela \textbf{analytics dashboard} os resultados de tais alterações.

\subsection{Motivação de Disciplinas}

A ideia inicial deste projeto foi concebida na aula de Roteiro para Jogos da Professora Erika Caramello, em que em uma das das atividades foi pedido para realizar um roteiro técnico, este que foi base de história e inspiração deste jogo. 

Foi aproveitada também uma das atividades da matéria do professor Claudemir, que consiste em desenvolver uma animação curta. Fora feita então um trailer do jogo, o que foi de ajuda para trazer ideias e adicionar conteúdo novo ao projeto.

Foram trazidas também referências de conteúdo extraídas da matéria do professor Miguel Saad. Mais em específico, em como desenvolver um \textit{flow} na narrativa do jogo.


%  - Em roteiro da professora Erika Caramello nasce o projeto como um roteiro.
%  - Aproveitamos a aula de Animação e Sons do professor Claudemir para forjar o trailer do jogo
%  - Utilizamos conhecimentos adquiridos na aula do professor Miguel, e para incluir Mihaly.


\subsection{Dificuldades Encontradas}

Uma das grandes dificuldades do grupo foi em gerenciar o tempo, inclusive em gerenciar o tempo para conseguir fazer reuniões com o Professor e Orientador Miguel Saad, pois todos tinham uma curta janela entre trabalho e aulas para conseguir organizar uma reunião mais formal.

Como o grupo tem um perfil muito técnico, preencher os tópicos da dissertação com o que se espera, um texto levou várias iterações para ser definido como pronto, o que acarretou em uma produção mais lenta do textos.

Como o projeto nasce antes de ser projeto de jogo e tem inspirações de folclores endêmicos de diversas partes do globo, o grupo teve dificuldades em achar referências boas e acessíveis, encontrando muita informação em sites e blogs informais, ainda assim uma grande bibliografia foi consultada, e nem todas as obras consultadas foram referenciadas.

% Algumas das dificuldades encontradas foram em organizar tópicos e seções da dissertação e levantar referências.

% Além disso, por questão de agenda e cronograma, houve certa dificuldade em realizar reuniões e encontros com o professor orientador Miguel.

% - Organização
% - Redigir
% - Encontros como o orientador por conta de agenda
% - Encontrar referências 


% Descreva e justifique para quais plataformas o jogo está sendo produzido (PC, Android, consoles etc).

% \section{Requerimentos de Rede}

% Descreva de que forma o jogo fará uso da Internet, caso utilize.
