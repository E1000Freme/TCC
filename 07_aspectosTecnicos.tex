
\section{Plataforma de Produção}

O jogo será produzido para PC, por ter maiores opções de interações assim como um maior processamento, nos permitindo maior permeabilidade com o 3D.
% Descreva e justifique para quais plataformas o jogo está sendo produzido (PC, Android, consoles etc).


\section{Hardware e Software de Desenvolvimento}

O jogo será produzido em Unity3D devido ao time já ter pré conhecimento na ferramenta e na linguagem de programação(C\#) utilizada por ela. Além disso, tem sido um dos motores que mais crescem tanto no mercado nacional como internacional.

Para a modelagem será utilizado Maya e Blender, por familiaridade do time.

Para o desenvolvimento de imagens de conceito, será utilizado Gravit Designer, sendo uma boa opção para gráficos vetoriais, além de ser gratuita.

\section{Controle de Acesso}

O código do jogo será distribuído para os desenvolvedores através da ferramenta de versionamento Git. Além disso, por conta o tamanho e quantidade de arquivos, a seção de assets do jogo será distribuída numa ferramenta a parte, armazenado na nuvem.

\section{Testes}
\subsection{Software}

Os testes do jogo serão realizados dentro da própria ferramenta de desenvolvimento, numa rotina de realizar alterações e logo após, testar atrás de erros.

\subsection{Usabilidade}

Serão realizados testes práticos, sendo executados logo que hajam alterações no programa

% Descreva e justifique para quais plataformas o jogo está sendo produzido (PC, Android, consoles etc).

% \section{Requerimentos de Rede}

% Descreva de que forma o jogo fará uso da Internet, caso utilize.
